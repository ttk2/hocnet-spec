\documentclass[11pt]{article}
\usepackage[linktocpage=true]{hyperref}
\usepackage[utf8]{inputenc}

\title{\textbf{Hocnet a payment based mesh network protocol built on Batman-Adv}}
\date{2016-12-10}
\begin{document}

\maketitle

\section{Work in progress}

\tableofcontents

\section{Abstract}

The largest challenge in providing internet services to both already connected and emerging internet populations is the so called 'last mile`. With the explosion number and dramatic decrease in cost of connected devices ad-hoc networks and by extension mesh networks have looked increasingly attractive as solutions to connectivity problems.

In this paper we explore the possibility of a for-profit mesh service for participation and use of the general population as a last-mile internet service provider with infrastructure created from a dynamic mix of of semi specialized consumer hardware and professional equipment inserted at key points.

\section{Feasibility of mesh networking as a primary ISP}

    \subsection{Participation}

    WiFi enabled devices are ubiquitous and often spend a significant amount of their active time idle, the potential of using a network of normal consumer devices to replace expensive, difficult to construct, and often monopolistic last mile infrastructure is enormous.

    The challenge is both motivation and enablement for the average consumer to participate in such a network. We propose a protocol in which a node will route data to an exit gateway and each node along this path will be paid at a rate of that nodes own choosing. In later sections we will examine the techniques for providing such a payment infrastructure, in this section we focus only on the feasibility of the network itself. Feasibility being defined as reasonable cost, connection speed, and complexity for a network base on mostly consumer hardware, with the possibility of professional hardware being profitable at network choke points being left open.

    \subsection{Network throughput}

    Throughput in wireless mesh networks is limited both by packet loss and channel availability. While each individual node may be capable of several hundred megabits per second of wireless throughput actual throughput is cut in half immediately if the transmission must be rebroadcast on the same channel. Having two devices operating on different channels can mitigate this issue at the cost of compounding the problem of available channels.

    \textbf{(TODO the below is very rough, missing citations and possibly misleading)}

    There are three non-overlapping 2.4ghz WiFi channels available in the United States and 6ish channels with varying bandwidths available in the 5ghz spectrum. While other channels are available (most interestingly high power wireless A for rural users and 60ghz wireless AD for ultra dense urban environments) we will restrict ourselves to these bands as they are the most common. Many of the available 5ghz channels require DFS compliance to be used and are ignored by most consumer hardware. As this is a software issue more than a hardware one we will make an assumption in our favor there.

    At 300mbps single direction maximum per 2.4ghz N channel and 800mbps single direction maximum per 5ghz AC channel. This is a grand total of 4.8Gbps per 5ghz signal range and 900Mbps per 2.4ghz signal range, in this case smaller signal ranges due to walls or other physical interference work both for and against the network. By isolating nodes total bandwidth available to each node increases, but the number of potential hops to reach an internet gateway increases.

    Batman-Adv has a maximum of three retires for wireless interfaces in its ARQ scheme. (mix of worst case and best case estimates, maybe move everything to average case? Insufficient data for average case)

    \subsection{Network latency}

    (determined by number of hops and network coding for batched packets, which might only apply to OGM's)

    \subsection{Cost of operation}

    (power consumption of a node use to estimate floor price of a hop, this combined with avg number of hops to gateway estimate should determine floor price of bandwidth, should be cost feasible by a mile)

\section{Modifications to Batman-Adv protocol spec}

	To understand this section you should be familiar with the BATMAN protocol in broad strokes \cite{batman}. It's a short and very accessible paper, please read it if you have not already.

	While the BATMAN protocol has significant inherent overhead this is addressable through optimization \cite{catwoman, batroam} and it provides real world performance comparable to more conceptually complex protocols \cite{meshperf}. This allows for the addition of payment in a relatively straightforward manner.

	\subsection{Addition of cost criteria}

		Specifically we propose a new field be added to BATMAN originator messages (henceforth refereed to as OGMs), this field is simply a half-precision floating point value for the cost of transmitting a pre-arranged number of packets in a single direction, an originator creating a new OGM for broadcast into the network would initialize this value to their 'hop cost'. A node rebroadcasting a received OGM would update this field by adding their own hop cost to the existing value before forwarding.
	The process for selecting which OGM to retain and rebroadcast is likewise updated to account for the cost field. Instead of choosing the best OGM using the TQ field as the sole criteria the ratio of cost to TQ is chosen.

	\subsection{Payment chains}

        In real world conditions route changes in BATMAN and other mesh networks can occur very frequently, on the order of once every several seconds, while it is possible to reduce the amount of "route flipping" through careful protocol changes route flipping represents a fundamentally different challenge to mesh network with payment \cite{meshflip}.

	\subsection{Microdebt and trust}


	\subsection{Protocol changes}

	\subsection{Proof of correctness}

	    \subsubsection{Assuming no malicious nodes}

	    \subsubsection{Assuming unorganized malicious nodes}

	    \subsubsection{Assuming malicious collusion}


\section{Payment models}

    \subsection{Cryptocurrency}

    \subsection{Semi centralized cryptocurrency}

    \subsection{Semi centralized traditional currency}


\begin{thebibliography}{1}



\bibitem{batman}
David Johnson, Ntsibane Ntlatlapa, and Corinna Aichele.
\textit{A simple pragmatic approach to mesh routing using BATMAN} (2008)

\bibitem{catwoman}
Cigno, R., and Daniele Furlan.
\textit{Improving BATMAN Routing Stability and Performance} (2011)

\bibitem{batroam}
Quartulli, Antonio, and Renato Lo Cigno.
\textit{Client announcement and Fast roaming in a Layer-2 mesh network} (2011)

\bibitem{meshperf}
Murray, David, Michael Dixon, and Terry Koziniec.
\textit{An experimental comparison of routing protocols in multi hop ad hoc networks} Telecommunication Networks and Applications Conference (ATNAC), 2010 Australasian. IEEE, 2010.

\bibitem{meshflip}
Britton, Matthew, and Andrew Coyle.
\textit{Performance analysis of the BATMAN wireless ad-hoc network routing protocol with mobility and directional antennas} Military Communications and Information Systems Conference (MilCIS), 2011. IEEE, 2011.

\end{thebibliography}
>>>>>>> Added some citations

\end{document}
